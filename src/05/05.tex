\documentclass[11pt]{beamer}
\usepackage[UTF8]{ctex}
%\usepackage[utf8]{inputenc}
%\usepackage[T1]{fontenc}
%\usepackage{lmodern}
%\usepackage{amsmath}
%\usepackage{amsfonts}
%\usepackage{amssymb}
\usepackage{graphicx}
\usetheme{CambridgeUS}

%%% 超链接
\usepackage[]{hyperref}

\usepackage{pdfpages}

%%%%%  代码
\usepackage{listings}\usepackage{color}  
\definecolor{dkgreen}{rgb}{0,0.6,0}  
\definecolor{gray}{rgb}{0.5,0.5,0.5}  
\definecolor{mauve}{rgb}{0.58,0,0.82}  

\lstset{ %  
	basicstyle=\ttfamily,
	%basicstyle=\footnotesize,           % the size of the fonts that are used for the code  
	numbers=left,                   % where to put the line-numbers  
	numberstyle=\tiny\color{gray},  % the style that is used for the line-numbers  
	stepnumber=1,                   % the step between two line-numbers. If it's 1, each line   
	% will be numbered  
	numbersep=5pt,                  % how far the line-numbers are from the code  
	backgroundcolor=\color{white},      % choose the background color. You must add \usepackage{color}  
	showspaces=false,               % show spaces adding particular underscores  
	showstringspaces=false,         % underline spaces within strings  
	showtabs=false,                 % show tabs within strings adding particular underscores  
	frame=single,                   % adds a frame around the code  
	rulecolor=\color{black},        % if not set, the frame-color may be changed on line-breaks within not-black text (e.g. commens (green here))  
	tabsize=2,                      % sets default tabsize to 2 spaces  
	captionpos=b,                   % sets the caption-position to bottom  
	breaklines=true,                % sets automatic line breaking  
	breakatwhitespace=false,        % sets if automatic breaks should only happen at whitespace  
	% title=\lstname,                   % show the filename of files included with \lstinputlisting;  
	% also try caption instead of title  
	keywordstyle=\color{blue},          % keyword style  
	commentstyle=\color{dkgreen},       % comment style  
	stringstyle=\color{mauve},         % string literal style  
	escapeinside={\%*}{*)},            % if you want to add LaTeX within your code  
	morekeywords={*,...}               % if you want to add more keywords to the set  
}  

% 分栏-用于目录
\usepackage{multicol}


%%%%%表格
\usepackage{longtable}
\usepackage{subfigure}
\usepackage{booktabs}

%%%%%
%\usepackage[backend=bibtex,sorting=none]{biblatex}
%\addbibresource{reference.bib} %BibTeX数据文件及位置
%\setbeamerfont{footnote}{size=\tiny} 
%\setbeamertemplate{bibliography item}[text]

%%% 参考文献
\usepackage{natbib}

%% LOGO放右上
\setbeamertemplate{frametitle}
{\begin{beamercolorbox}[wd=\paperwidth]{frametitle}
		\strut\hspace{0.5em}\insertframetitle\strut
		\hfill
		\raisebox{-2mm}{\includegraphics[width=1cm]{figures/HNUC.jpeg}}
	\end{beamercolorbox}
}

%%% 课件属性定义
\author{郭泰彪}
\title{区块链原理及应用}
\subtitle{第5课:区块链密码学基础(实验)}
\institute[湖工商大数据研究院]{湖南工商大学大数据与互联网创新研究院}
\date{2020年10月10日}
\titlegraphic{\includegraphics[width=2cm]{figures/HNUC.jpeg}}


\begin{document}
\begin{frame}[plain]
	\maketitle
\end{frame}

\begin{frame}
	\frametitle{目录}
	\begin{multicols}{2}
	\tableofcontents[sectionstyle=show,subsectionstyle=show/shaded]
	\end{multicols}
\end{frame}

\section{区块链密码学基础回顾}

\begin{frame}{哈希( hash)函数}
	内容... % TODO
\end{frame}

\section{实验任务介绍}
% 实现一个简单区块链,其中每个区块都指向上一个区块。
%这是一个区块链的基本的实现,不包含工作证明或点对点等高级功能。区块链功能的基本实现,可以通过动作集对区块链进行操作。
%运行成功后按照提示进行操作即可。

\begin{frame}{实验任务}
	实现简单区块链的任务说明:
	\begin{itemize}
		\item 包括:
		\begin{itemize}
			\item 将消息按照顺序打包成区块;
			\item 将区块按照顺序链接成区块链;
				\item 能够验证区块链是否被破坏:
			\begin{itemize}
				\item 能够验证消息是否被破坏;
			\item 能够验证区块是否被破坏;
			\end{itemize}	
		\item 命令行的操作界面;
		\end{itemize}
	\item 不包括:
	\begin{itemize}
		\item 不包含数字签名等认证功能等实现;
		\item 工作量证明等共识算法的实现;
		\item 点对点通信等分布式系统的实现;
	\end{itemize}
	\end{itemize}
\end{frame}

\begin{frame}{任务分析-消息Message}
	消息(Message)是用户存储到区块链的信息,
	\begin{itemize}
		\item 功能:
		\begin{itemize}
			\item :
		\end{itemize}
		\item 属性:
		\item 方法:
		\begin{itemize}
			\item  链接(link)
			\item 生成消息哈希(seal):			
				$hash_{message}=sha256(hash_{prev} + hash_{payload})$
		\end{itemize}
	\end{itemize}
\end{frame}

\section{实验环境的准备}

% 20分钟
\section{Python语言速览}
\subsection{人生苦短,我用Python}
\begin{frame}{人生苦短,我用Python}
	Python简洁却强大、简单却专业,它是当今世界最受欢迎的编程语言,学好它终身受用。\footnote{http://www.icourse163.org/course/BIT-268001,国家精品课程:Python语言程序设计}
	\begin{multicols}{2}
			\begin{itemize}
			\item	Python 是流行语言
			\item	Python 是入门语言
			\item	Python 是玩具语言
			\item	Python 是胶水语言
			\item	Python 是黑客语言
			\item	Python 是通用语言
		\end{itemize}
	\end{multicols}
\end{frame}

\begin{frame}{Python应用领域}
	\begin{figure}
		\centering
		\includegraphics[width=0.8\linewidth]{figures/python-field}
		\label{fig:python-field}
	\end{figure}
\end{frame}

\begin{frame}[allowframebreaks]{\#潘石屹用Python解决100个问题\#}
	微博话题:\textbf{\#潘石屹用Python解决100个问题\#}\cite{panshiyi100}
	
	2019年,当地产大佬潘石屹要把学习Python作为生日礼物送给自己的时候,微博上还多是一阵调侃之声,2020年潘同学已经在Python考试中得到99分的好成绩。时年56岁的小潘同学要再一次抓住“青春”的尾巴。
	\begin{figure}
		\centering
		\includegraphics[width=0.7\linewidth]{figures/pythonPanshiyi01}
		\label{fig:pythonpanshiyi01}
	\end{figure}

	\begin{figure}
	\centering
	\includegraphics[width=0.8\linewidth]{figures/pythonPanshiyi03}
	\label{fig:pythonpanshiyi03}
\end{figure}
\end{frame}

\begin{frame}{潘石屹谈为什么学Python}
	\begin{figure}
		\centering
		\includegraphics[width=0.6\linewidth]{figures/pythonPanshiyi02}
		\label{fig:pythonpanshiyi02}
	\end{figure}
\end{frame}

\subsection{代码格式}

% lstlisting报错问题解决思路:http://blog.sina.com.cn/s/blog_5e16f1770102dxps.html
\begin{frame}[fragile]
	\frametitle{代码格式}

		\begin{minipage}[t]{0.5\linewidth}
				\begin{itemize}
					\item Python是一门脚本语言,可以一行行输入并立即执行;
					\item Python也可以将代码保存到文件中,以\textbf{类}、\textbf{函数}等方式对代码进行组织。
				\end{itemize}
	\end{minipage}%
% 此行用于空格
	\begin{minipage}[t]{0.05\linewidth}
		\quad
\end{minipage}%
	\begin{minipage}[t]{0.4\linewidth}
		\begin{figure}
				\lstinputlisting[language=Python,basicstyle=\scriptsize ,title=python代码基本结构\cite{rossum1995python}]{./figures/basicpython.py}	
		\end{figure}
	\end{minipage}%
\end{frame}

\subsection{Python的基本语法}
\begin{frame}[fragile]
\frametitle{类}
\begin{minipage}[t]{0.5\linewidth}
	\begin{itemize}
	\item 类可以用来对代码进行组织和管理;
	\item 类可以对现实世界的对象进行抽象和建模;
	\item 类的基本组成包含属性和方法;
	\item 类支持继承等面向对象编程的功能;
\end{itemize}

例子
\begin{itemize}
	\item 类名: 人类(human)
	\item 属性:体重 (weight),年龄 (age),性别(sex);
	\item 方法:吃(eat), 睡(sleep), 玩(play);
\end{itemize}
\end{minipage}%
% 此行用于空格
\begin{minipage}[t]{0.05\linewidth}
	\quad
\end{minipage}%
	\begin{minipage}[t]{0.4\linewidth}
	\begin{figure}
		\lstinputlisting[language=Python,basicstyle=\scriptsize ,title=python定义人类]{./figures/classhuman.py}	
	\end{figure}
\end{minipage}
\end{frame}

\begin{frame}[fragile]
	\frametitle{函数}
	\begin{minipage}[t]{0.4\linewidth}
	\begin{itemize}
	\item 不属于任何类的"方法",没有self参数
	\item 函数内包含具体逻辑
\end{itemize}
	\end{minipage}%
	\begin{minipage}[t]{0.05\linewidth}
		\quad 	% 此行用于空格
	\end{minipage}%
	\begin{minipage}[t]{0.5\linewidth}
	\lstinputlisting[language=Python,basicstyle=\scriptsize,linerange={1-4}]{./figures/func.py}	
	
	\lstinputlisting[language=Python,basicstyle=\scriptsize,linerange={6-17}]{./figures/func.py}	
	\end{minipage}
\end{frame}

\begin{frame}[fragile]
	\frametitle{具体逻辑-(Python内置类+用户定义)}
	% 此行用于空格
	\begin{minipage}[t]{0.25\linewidth}
		判断语句:
		\begin{itemize}
			\item 条件语句
			\item 循环语句
		\end{itemize}
	\end{minipage}%
	\begin{minipage}[t]{0.25\linewidth}
		运算符:
		\begin{itemize}
			\item 算术运算符
			\item  比较运算符
			\item 赋值运算符
			\item 逻辑运算符
			\item ...
		\end{itemize}
	\end{minipage}%
	\begin{minipage}[t]{0.25\linewidth}
复杂数据结构:
\begin{itemize}
	\item 字符串
	\item 列表
	\item 元组
	\item 字典
\end{itemize}
\end{minipage}%
	\begin{minipage}[t]{0.25\linewidth}
	用户定义:
	\begin{itemize}
		\item 变量
		\item 自定义函数
		\item 自定义类
	\end{itemize}
\end{minipage}
\end{frame}

\begin{frame}{关键字}
	\scriptsize
	\begin{multicols}{3}
		\textbf{and}	逻辑运算符。
		
		\textbf{as}	创建别名。
		
		\textbf{assert}	用于调试。
		
	\textbf{break}	跳出循环。
		
		\textbf{class}	定义类。
		
		\textbf{continue}	继续循环的下一个迭代。
		
		\textbf{def}	定义函数。
		
		\textbf{del}	删除对象。
		
		\textbf{elif}	在条件语句中使用,等同于 else if。
		
		\textbf{else}	用于条件语句。
		
		\textbf{except}	处理异常,发生异常时如何执行。
		
		\textbf{False}	布尔值,比较运算的结果。
		
		\textbf{finally}	处理异常异常,都将执行一段代码。
		
		\textbf{for}	创建 for 循环。
		
		\textbf{from}	导入模块的特定部分。
		
		\textbf{global}	声明全局变量。
		
		\textbf{if}	写一个条件语句。
		
		\textbf{import}	导入模块。
		
		\textbf{in}	检查列表、元组等集合中是否存在某个值。
		
		\textbf{is}	测试两个变量是否相等。
		
		\textbf{lambda}	创建匿名函数。
		
		\textbf{None}	表示 null 值。
		
		\textbf{nonlocal}	声明非局部变量。
		
		\textbf{not}	逻辑运算符。
		
		\textbf{or}	逻辑运算符。
		
		\textbf{pass}	null 语句,一条什么都不做的语句。
		
		\textbf{raise}	产生异常。
		
		\textbf{return}	退出函数并返回值。
		
		\textbf{True}	布尔值,比较运算的结果。
		
		\textbf{try}	编写 try...except 语句。
		
		\textbf{while}	创建 while 循环。
		
		\textbf{with}	用于简化异常处理。
		
		\textbf{yield}	结束函数,返回生成器。
\end{multicols}
\end{frame}

\begin{frame}[fragile]
	\frametitle{判断语句}
	\begin{minipage}[t]{0.5\linewidth}
		\begin{itemize}
			\item	条件语句:
			\begin{itemize}
				\item if else判断
			\end{itemize}
			\item 循环语句:
			\begin{itemize}
				\item while循环
				\item for循环
				\item 中断循环标志:break,continue,pass
			\end{itemize}
		\end{itemize}
	\end{minipage}%
	% 此行用于空格
	\begin{minipage}[t]{0.05\linewidth}
		\quad
	\end{minipage}%
	\begin{minipage}[t]{0.4\linewidth}
		\begin{figure}
			\lstinputlisting[language=Python,basicstyle=\scriptsize ,title=python定义人类]{./figures/classhuman.py}	
		\end{figure}
	\end{minipage}
\end{frame}

\begin{frame}[fragile]
	\frametitle{运算符}
	\begin{minipage}[t]{0.33\linewidth}
		算数运算符:
		
	\begin{tabular}{|c|c|}
		\hline 
	+	&  简单赋值\\ 
		\hline 
	-	&加法赋值  \\ 
		\hline 
	*	& 减法赋值 \\ 
		\hline 
	/	& 乘法赋值 \\ 
		\hline 
	\%	&  除法赋值\\ 
		\hline 
	**=	&取模赋值  \\ 
		\hline 
	//	&  幂赋值\\ 
		\hline 
	\end{tabular} 
	\end{minipage}%
	\begin{minipage}[t]{0.33\linewidth}
	赋值运算符:
	
	\begin{tabular}{|c|c|}
		\hline 
		=	&  简单赋值\\ 
		\hline 
		+=	&加法赋值  \\ 
		\hline 
		-=	& 减法赋值 \\ 
		\hline 
		*=	& 乘法赋值 \\ 
		\hline 
		/=	&  除法赋值\\ 
		\hline 
		\%=	&取模赋值  \\ 
		\hline 
		**=	&  幂赋值\\ 
		\hline 
		//=	& 取整除赋值 \\ 
		\hline 
	\end{tabular} 
\end{minipage}%
	\begin{minipage}[t]{0.33\linewidth}
		比较运算符:
		
	\begin{tabular}{|c|c|}
	\hline 
	==	&  等于\\ 
	\hline 
	!=	&不等于  \\ 
	\hline 
	>	& 大于 \\ 
	\hline 
	<	&小于  \\ 
	\hline 
	>=	&  大于等于\\ 
	\hline 
	<=	&  小于等于\\ 
	\hline 
\end{tabular} 
	\end{minipage}
\end{frame}

\begin{frame}[fragile]
	\frametitle{复杂数据结构}
	\begin{minipage}[t]{0.5\linewidth}
		\begin{itemize}
			\item	dosomething
		\end{itemize}
	\end{minipage}%
	% 此行用于空格
	\begin{minipage}[t]{0.05\linewidth}
		\quad
	\end{minipage}%
	\begin{minipage}[t]{0.4\linewidth}
		\begin{figure}
			\lstinputlisting[language=Python,basicstyle=\scriptsize ,title=python定义人类]{./figures/classhuman.py}	
		\end{figure}
	\end{minipage}
\end{frame}

\begin{frame}[fragile]
	\frametitle{title}
	\begin{minipage}[t]{0.5\linewidth}
		\begin{itemize}
		\item	dosomething
		\end{itemize}
	\end{minipage}%
	% 此行用于空格
	\begin{minipage}[t]{0.05\linewidth}
		\quad
	\end{minipage}%
	\begin{minipage}[t]{0.4\linewidth}
		\begin{figure}
			\lstinputlisting[language=Python,basicstyle=\scriptsize ,title=python定义人类]{./figures/classhuman.py}	
		\end{figure}
	\end{minipage}
\end{frame}
\subsection{更多Python资料}
\begin{frame}[fragile]
	\frametitle{更多Python资料}
	
\begin{itemize}
		\item 	视频教程:
		\begin{itemize}
			\item \href{http://www.icourse163.org/course/BIT-268001}{\underline{北理工Python程序设计系列}}
		\end{itemize}

		\item 	书籍:
		\begin{itemize}
			\item \href{https://item.jd.com/12279949.html}{\underline{Python基础教程(第三版)}}
		\end{itemize}

		\item 	开发工具:
		\begin{itemize}
			\item \href{https://www.jetbrains.com/pycharm/}{\underline{PyCharm}}:一个智能化的Python 开发环境
			\item \href{https://www.anaconda.com/products/individual}{\underline{conda}} :集成了很多Python数据分析工具的Python解决方案
		\end{itemize}

		\item 	Python项目:
		\begin{itemize}
			\item \href{https://github.com/vinta/awesome-python}{\underline{awesome-python}}:包含众多优秀Python项目
		\end{itemize}
\end{itemize}

{\small 点击可以\textbf{资料名称}可以直达下载地址}。
\end{frame}

\section{简单区块链的定义和实现}
\subsection{区块的定义和实现}

\subsection{链的定义和实现}

\subsection{消息的定义和实现}

\subsection{业务逻辑的定义和实现}

\subsection{运行}

\section{小结}

\section{附录}
\begin{frame}[allowframebreaks]{附录1:简单区块链源代码}
	\lstinputlisting[language=Python,basicstyle=\tiny,title="简单区块链源代码\cite{simplechaingithub}"]{./figures/simple_chain.py}
\end{frame}

\section{参考文献}
\begin{frame}{参考资料}
%	\printbibliography
\bibliographystyle{plain}
\bibliography{reference}
\end{frame}

\section*{谢谢聆听}

\begin{frame}
	\begin{minipage}[t]{0.5\linewidth}
		\begin{center}
			\begin{figure}
				\vspace{10pt}
				
				{\Huge 谢谢聆听}
				
				\vspace{30pt}
				郭泰彪
				
				\vspace{10pt}
				{\tiny 湖南工商大学大数据与互联网创新研究院}
			\end{figure}
			\begin{figure}
				
			\end{figure}
		\end{center}
	\end{minipage}%
	\begin{minipage}[t]{0.4\linewidth}
		\begin{figure}
			\centering
			\texttt{blockchain101}
			
			\includegraphics[width=0.6\linewidth]{figures/blockchain101qrcode}
			
			{\footnotesize \texttt{Star| Fork| Issue}}
		\end{figure}
	\end{minipage}%
\end{frame}

\end{document}